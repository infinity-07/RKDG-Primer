\usepackage{ctex} % 排版中文文章
\usepackage{physics}
\usepackage{booktabs,tabularx,multicol,multirow} % 制作表格
\usepackage{appendix}
\usepackage{amsmath,amsthm,amssymb,amsfonts}    % 数学符号与字体

\usepackage{float}

\usepackage{hyperref}   % 添加超链接
\hypersetup{hidelinks}  % 把超链接周围的红框隐藏起来
\usepackage[dvipsnames]{xcolor} % 在latex中使用颜色

\usepackage{subfig,graphicx}    % 插入照片
\usepackage{enumitem}
\usepackage{lipsum,zhlipsum} %生成一些测试文本

\usepackage[left=2.0cm, right=2.0cm, top=2.5cm, bottom=2.5cm]{geometry} % 调整页边距
\usepackage{soul} % 高亮文本

% 字符编码和数学宏包
%\usepackage[utf8]{inputenc}
\usepackage{amsmath, amsthm, amsfonts, amssymb}
\usepackage{tikz} %很强大的绘图宏包

%自定义设置字体
\usepackage{fontenc}
\setmainfont{Times New Roman} % 设置英文衬线字体
%\setsansfont{Arial} % 设置英文无衬线字体
%\setmonofont{Courier New} % 设置英文等宽字体

% 表格宏包
\usepackage{multirow, booktabs} %两种不同处理表格,取决于不同情况的处理

% 页面和排版宏包
\usepackage{fullpage} %
\usepackage{enumitem} %处理列表(包括有序、无序两类)
\usepackage{fancyhdr} %
\usepackage{fontawesome}
\usepackage{geometry} %设置页面边距

% ============== 设置一个好看的页眉页脚款式 ===============
\geometry{ % 设置页面布局
  includeheadfoot, % 将页眉和页脚计算在页面范围内
  left=1in, % 设置左边距
  right=1in, % 设置右边距
  top=1in, % 设置上边距
  bottom=1in, % 设置下边距
}


% 其他页面设置
\geometry{margin=1in, headsep=0.75in}
% ====================================================

% 数学字体宏包
\usepackage{mathrsfs} %比如\mathcal,\mathscr等


% 图片和文本宏包
\usepackage{wrapfig} %处理浮动图片
\usepackage{graphicx}
\usepackage{caption}
\usepackage{setspace,calc} %前者设置距离,后者设置文本长度
\usepackage{multicol} %分列处理文本

% 公式宏包
\usepackage{cancel} %引入划去线
\usepackage[retainorgcmds]{IEEEtrantools} %表达式排版,比如align等 
\usepackage{empheq} %创建带有框和标签的数学公式

% 框架和颜色宏包
\usepackage{framed} %边框
\usepackage[most]{tcolorbox}
\colorlet{shadecolor}{orange!15} %颜色自定义

% ======= 定理环境边框 ======
\definecolor{browna}{rgb}{0.76,0.72,0.65}
\definecolor{brownb}{rgb}{0.71,0.69,0.65}
\definecolor{blacka}{RGB}{245,245,245}





%%%%%%%%%%%%%%%%%%%
\newtheorem{remark}{Remark}
\newtheorem{example}{Example}

% \newtcbtheorem[number within=section]{example}{Example}{colback=OliveGreen!10,colframe=Green!70,fonttitle=\bfseries}{example}






\newtcbtheorem[number within=subsection]{theorem}{Theorem}{
  enhanced,
  sharp corners,
  attach boxed title to top left={
      yshifttext=-1mm
    },
  colback=white,
  colframe=brownb,
  fonttitle={\bfseries},
  fontupper={\kaishu\itshape},
  boxed title style={
      sharp corners,
      size=small,
      colback=brownb,
      colframe=brownb,
    },
  breakable  % 添加这一行以启用自动断页
}{thm}


\newtcbtheorem[number within=subsection]{lem}{Lemma}{
  enhanced,
  sharp corners,
  attach boxed title to top left={
      yshifttext=-1mm
    },
  colback=white,
  colframe=brownb,
  fonttitle=\bfseries,
  boxed title style={
      sharp corners,
      size=small,
      colback=brownb,
      colframe=brownb,
    },
  breakable  % 添加这一行以启用自动断页 
}{lem}

\newtcbtheorem[number within=subsection]{cor}{Corollary}{
  enhanced,
  sharp corners,
  attach boxed title to top left={
      yshifttext=-1mm
    },
  colback=white,
  colframe=brownb,
  fonttitle=\bfseries,
  boxed title style={
      sharp corners,
      size=small,
      colback=brownb,
      colframe=brownb,
    },
  breakable  % 添加这一行以启用自动断页 
}{cor}

\newtcbtheorem[number within=subsection]{definition}{Definition}{
  enhanced,
  sharp corners,
  attach boxed title to top left={
      yshifttext=-1mm
    },
  colback=white,
  colframe=browna,
  fonttitle=\bfseries,
  boxed title style={
      sharp corners,
      size=small,
      colback=browna,
      colframe=browna,
    },
  breakable  % 添加这一行以启用自动断页 
}{defn}






% ========= 标题、作者配置 =====


\usepackage{titling} % 自定义标题


% ============================

\usepackage{hyperref}