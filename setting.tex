\usepackage{ctex} % 排版中文文章
\usepackage{physics}
\usepackage{booktabs,tabularx,multicol,multirow} % 制作表格
\usepackage{appendix}
\usepackage{amsmath,amsthm,amssymb,amsfonts}    % 数学符号与字体

\usepackage{hyperref}   % 添加超链接
\hypersetup{hidelinks}  % 把超链接周围的红框隐藏起来

\usepackage{subfig,graphicx}    % 插入照片
\usepackage{enumitem}
\usepackage{lipsum,zhlipsum} %生成一些测试文本

\usepackage[left=2.0cm, right=2.0cm, top=2.5cm, bottom=2.5cm]{geometry} % 调整页边距
\usepackage{soul} % 高亮文本

% 字符编码和数学宏包
%\usepackage[utf8]{inputenc}
\usepackage{amsmath, amsthm, amsfonts, amssymb}
\usepackage{tikz} %很强大的绘图宏包

%自定义设置字体
\usepackage{fontenc}
\setmainfont{Times New Roman} % 设置英文衬线字体
%\setsansfont{Arial} % 设置英文无衬线字体
%\setmonofont{Courier New} % 设置英文等宽字体


%设置正文文本分段空行不置顶
\usepackage{indentfirst} % 导入indentfirst包
\setlength{\parindent}{0pt} % 取消首行缩进
\setlength{\parskip}{\baselineskip} % 段落之间空一行

% 表格宏包
\usepackage{multirow, booktabs} %两种不同处理表格,取决于不同情况的处理

% 页面和排版宏包
\usepackage{fullpage} %
\usepackage{enumitem} %处理列表(包括有序、无序两类)
\usepackage{fancyhdr} %
\usepackage{fontawesome}
\usepackage{geometry} %设置页面边距

% ============== 设置一个好看的页眉页脚款式 ===============
\geometry{ % 设置页面布局
  includeheadfoot, % 将页眉和页脚计算在页面范围内
  left=1in, % 设置左边距
  right=1in, % 设置右边距
  top=1in, % 设置上边距
  bottom=1in, % 设置下边距
}

\pagestyle{fancy}
\fancyhead{} % 清空页眉
\fancyfoot{} % 清空页脚
\setlength{\headheight}{12.20pt}
\definecolor{structurecolor}{RGB}{68,79,173}
\fancyhead[L]{\color{structurecolor}\kaishu {\faPaperPlaneO\ Academic year 2023 : Algebra}}
\fancyhead[R]{\color{structurecolor}\kaishu すうがく數學!}
\fancyfoot[L]{\color[RGB]{204,52,51}{\faLink}}
\fancyfoot[R]{\color[RGB]{0,64,116}{\faUnlink}}
\fancyfoot[C]{\color{structurecolor}{\faGg 第~\thepage ~页\faGg}}
\newcommand\ee{\mathrm{e}}

% 其他页面设置
\geometry{margin=1in, headsep=0.75in}
% ====================================================

% 数学字体宏包
\usepackage{mathrsfs} %比如\mathcal,\mathscr等


% 图片和文本宏包
\usepackage{wrapfig} %处理浮动图片
\usepackage{graphicx}
\usepackage{caption}
\usepackage{setspace,calc} %前者设置距离,后者设置文本长度
\usepackage{multicol} %分列处理文本

% 公式宏包
\usepackage{cancel} %引入划去线
\usepackage[retainorgcmds]{IEEEtrantools} %表达式排版,比如align等 
\usepackage{empheq} %创建带有框和标签的数学公式

% 框架和颜色宏包
\usepackage{framed} %边框
\usepackage{xcolor} %颜色
\usepackage[most]{tcolorbox}
\colorlet{shadecolor}{orange!15} %颜色自定义

% ======= 定理环境边框 ======


\definecolor{browna}{rgb}{0.76,0.72,0.65}
\definecolor{brownb}{rgb}{0.71,0.69,0.65}
\definecolor{blacka}{RGB}{245,245,245}

%eg环境的配置
% 注意行末需要把空格注释掉,不然画出来的方框会有空白竖线
\usepackage[strict]{changepage} % 提供一个 adjustwidth 环境

\newenvironment{eg}{%
  \def\FrameCommand{%
    \hspace{1pt}%
    {\color{brownb}\vrule width 2pt}%
    {\color{blacka}\vrule width 4pt}%
    \colorbox{blacka}%
  }%
  \MakeFramed{\advance\hsize-\width\FrameRestore}%
  \begin{adjustwidth}{}{7pt}%
    \vspace{-14pt}\textbf{Example}\newline\kaishu\itshape
    }{%
  \end{adjustwidth}\endMakeFramed%
}

%re环境配置
\newenvironment{re}{
  \par\medskip
  \textbf{Remark.}
  \begin{itshape}\kaishu
    }{
  \end{itshape}
  \par\medskip
}

%pf环境设置
\definecolor{Bloodred}{RGB}{136,0,21}

\newenvironment{pf}{
  \noindent\textcolor{Bloodred}{\emph{\ensuremath{\mathfrak{Proof.}}}}\vspace{-1em}
  \par
  %\color{gray}
  \kaishu
}{
  \hfill\textcolor{Bloodred}{$\blacksquare$}
}




%%%%%%%%%%%%%%%%%%%
\newtheorem{remark}{Remark}
\newtheorem{example}{Example}





\newtcbtheorem[number within=subsection]{theorem}{Theorem}{
  enhanced,
  sharp corners,
  attach boxed title to top left={
      yshifttext=-1mm
    },
  colback=white,
  colframe=brownb,
  fonttitle={\bfseries},
  fontupper={\kaishu\itshape},
  boxed title style={
      sharp corners,
      size=small,
      colback=brownb,
      colframe=brownb,
    },
  breakable  % 添加这一行以启用自动断页
}{thm}


\newtcbtheorem[number within=subsection]{lem}{Lemma}{
  enhanced,
  sharp corners,
  attach boxed title to top left={
      yshifttext=-1mm
    },
  colback=white,
  colframe=brownb,
  fonttitle=\bfseries,
  boxed title style={
      sharp corners,
      size=small,
      colback=brownb,
      colframe=brownb,
    },
  breakable  % 添加这一行以启用自动断页 
}{lem}

\newtcbtheorem[number within=subsection]{cor}{Corollary}{
  enhanced,
  sharp corners,
  attach boxed title to top left={
      yshifttext=-1mm
    },
  colback=white,
  colframe=brownb,
  fonttitle=\bfseries,
  boxed title style={
      sharp corners,
      size=small,
      colback=brownb,
      colframe=brownb,
    },
  breakable  % 添加这一行以启用自动断页 
}{cor}

\newtcbtheorem[number within=subsection]{definition}{Definition}{
  enhanced,
  sharp corners,
  attach boxed title to top left={
      yshifttext=-1mm
    },
  colback=white,
  colframe=browna,
  fonttitle=\bfseries,
  boxed title style={
      sharp corners,
      size=small,
      colback=browna,
      colframe=browna,
    },
  breakable  % 添加这一行以启用自动断页 
}{defn}

% \let\oldthm\thm
% \renewcommand{\theorem}[2][]{\oldthm[{#1}]{#2}{#1}}

% \let\oldlem\lem
% \renewcommand{\lem}[2][]{\oldlem[{#1}]{#2}{#1}}

% \let\oldcor\cor
% \renewcommand{\cor}[2][]{\oldcor[{#1}]{#2}{#1}}

% \let\olddefn\defn
% \renewcommand{definition}[2][]{\olddefn[{#1}]{#2}{#1}}

% ================


%--------------------代码块设置---------------------------%
%minted设置
\usepackage{minted}

\setminted[python]{
  style=emacs,%切换不同的风格
  %style=autumn, 
  linenos, % 添加行号
  breaklines=true,
  breakafter=\,
  breaksymbolleft=\raisebox{0.8ex}{\small\ensuremath{\hookleftarrow}}
}

\newenvironment{code}{%
  \begin{tcolorbox}[breakable=true,colback=white, colframe=black, title=Code, left=7mm]%
    }{%
  \end{tcolorbox}%
}


%行内代码样式设置
\newcommand{\textcode}[1]{\tikz[baseline=(char.base)]{
    \node[fill=gray!20,shape=rectangle,rounded corners,inner sep=2pt] (char) {\texttt{#1}};
  }}

%-------------------------------------------------------%








% ========= 标题、作者配置 =====


\usepackage{titling} % 自定义标题

% 自定义标题样式,使其水平垂直居中
%\renewcommand{\maketitle}{
%  \begin{center}
%    \vspace*{\fill}
%    \Huge\textbf{\thetitle} \\[1em]
%    \Large\theauthor \\[1em]
%    %\large\thedate
%    \vspace*{\fill}
%  \end{center}
%}


% ============================

\usepackage{hyperref}
%超链接样式设置
\hypersetup{
  colorlinks=true,
  linkcolor=blue,  % 设置内部链接的颜色为蓝色
  urlcolor=[RGB]{181,39,28},    % 设置外部链接的颜色为棕红色
}